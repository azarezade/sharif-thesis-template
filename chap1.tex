% !TEX encoding = UTF-8 Unicode
\chapter{مقدمه}\label{Chap:Chap1}

%==================================================================
تحلیل تاریخچه رویداد از مسایل کلاسیک آمار است که به مدل‌سازی، پیش‌بینی و   (اخیرا) کنترل رویداد‌ها می‌پردازد.
منظور از \trans{رویداد}{Event}، یک اتفاق خاص در یک زمان مشخص است؛ مانند بیماری، زلزله، مشاهده یک خبر یا رفتن به یک مکان.
هر رویداد با زمان وقوع (و نوع) آن مشخص می‌شود. 
 رویداد‌ها می‌توانند  از نوع \trans{بقا}{Survival} یا \trans{بازگشتی}{Recurrent} باشند...
\section{تحلیل تاریخچه رویداد}
برای مدل‌سازی رویدادها (پدیده‌هایی که به صورت اتفاق‌هایی‌ در زمان پیوسته رخ می‌دهند) از 
\trans{فرآیندهای تصادفی نقطه‌ای}{Stochastic point process} 
 استفاده می‌شود.
فرآیندهای نقطه‌ای، برای مدل‌سازی در گستره وسیعی از کاربردهای شبکه‌های اجتماعی و سیستم‌های اطلاعاتی مانند؛ انتشار اطلاعات~\cite{Rodriguez2011, du13nips, zhao2015seismic}، دینامک نظرات~\cite{de2016learning}، رقابت محصولات~\cite{Valera2015,zarezade2015correlated}، اتکاپذیری اطلاعات~\cite{reliability2017tabibian} یا یادگیری انسان‌ها~\cite{hdhp2017learning} استفاده شده است.

 یک فرآیند تصادفی را می‌توان با دانستن توزیع توام متناهی‌اش به طور کامل بیان کرد  (پیوست \ref{Chap:App1}، قضیه کولموگروف\LTRfootnote{Kolmogrov}).  
توزیع توام متناهی $f(t_1,t_2,\ldots,t_n)$،  چگالی احتمال وقوع رویدادها در زمان‌های $t_1$ تا $t_n$ دلخواه را بیان می‌کند.
طبق قانون بیز می‌توان این توزیع توام را به فرم زیر نوشت:
\begin{align}
	f(t_1,t_2,\cdots,t_n)=\prod_i f(t_i|t_{i-1:1})
\end{align}

%==================================================================
\section{هدف پژوهش}
هدف از این پژوهش، مدل‌سازی و کنترل رویداد‌ها با در نظر گرفتن فرضیات مساله و وجود ساختار شبکه است.
در واقع هر یک از گره‌های شبکه مجموعه‌ای از رویدادها تولید می‌کند، و در مساله مدل‌سازی به دنبال این هستیم که از فرضیات مساله و تعامل بین گره‌‌ها و تاریخچه رویدادهایشان برای تعیین رویداد بعدی هر کاربر در شبکه استفاده کنیم...

\section{نوآوری‌های رساله}
نوآوری‌‌های این رساله در بخش مدل‌سازی و کنترل رویداد شامل...

\section{ساختار رساله}
در ادامه این رساله ابتدا در فصل  \ref{Chap:Chap2}، به پژوهش‌های پیشین در زمینه مدل‌سازی و کنترل رویدادها می‌پردازیم. در فصل ...


